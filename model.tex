
\documentclass[12pt]{article}

\usepackage[english]{babel}
\usepackage[utf8x]{inputenc}
\usepackage{amsmath}
\usepackage{graphicx}

\title{PH Capstone Model}
\author{Columbia University}

\begin{document}
\maketitle

\begin{abstract}
Description of the model for the PH Capstone project. As designed by Mark Orr. translated to \LaTeX and tweaked by Anders Pearson.
\end{abstract}

\section{Introduction}

I'm rewriting stuff in \LaTeX partially to have a good text copy that can be kept up-to-date in the git repo, and partly to learn \LaTeX better.


\section{Core Agent Variables}

\begin{itemize}
\item mass
\item input
\item base-output
\item total-output
\item force-of-habit
\item c-control (for conscious control of energy intake)
\item physical-activity
\item friend-input
\item friend-output
\end{itemize}


\section{Environment-Variables}

All of these are (0-1), continuous variables

\begin{itemize}
\item food-exposure
\item food-energy-density
\item food-advertising
\item food-convenience
\item food-literacy
\item recreation-activity
\item domestic-activity
\item transport-activity
\item education-activity
\end{itemize}


\section{Agent Equations}

\subsection{Agent Mass}

\begin{equation}
mass = mass_{t-1} + f(\text{input} - \text{total-output}) * \gamma_1
\end{equation}

and $\gamma_1 = 1$. Where $f(\text{input} - \text{total-input})$ is

\begin{equation}
\frac{1}{1 + e^{-(input - \text{total-output})}} - 0.50
\end{equation}

and $\gamma_1 = 1$.


\subsection{Agent Energy Intake}

\begin{equation}
input = \text{total-output} + \text{force-of-habit} * \gamma_2 + \text{friend-input} * \gamma_3 - \text{c-control}* \gamma_4
\end{equation}

Where force-of-habit is

\begin{equation}
\frac{1}{1 + e^{-10(\sum_{1}^{4}FoodEnviron - 0.50)}}
\end{equation}

c-control is

\begin{equation}
\frac{1}{1 + e^{-10(\sum_{5}^{5}FoodEnviron - 0.50)}}
\end{equation}

friend-input is

\begin{equation}
\frac{1}{1 + e^{-10(x - 0.50)}} - 0.50
\end{equation}

where x is \% of social network ties with input $>=$ agent's own input. and $ \gamma_2 \ldots \gamma_4 = 1$.

\subsection{Agent Energy Expenditure}

\begin{equation}
\text{total-output} = \text{base-output} + \text{physical-activity}*\gamma_5 + \text{friend-output}*\gamma_6
\end{equation}

where  physical-activity is

\begin{equation}
\frac{1}{1 + e^{-10(\sum_{1}^{4}PhysActivEnviron - 0.50)}}
\end{equation}

and friend-output is

\begin{equation}
\frac{1}{1 + e^{-10(x - 0.50)}} - 0.50
\end{equation}


where x is \% of social network ties with total-output $>=$ agent's own input and $\gamma_5, \gamma_6 = 1$


\section{Dynamics of Model}

\subsection{Initialization of the model}

Note: $N(x,y)$ is normal distribution with mean x and sd y.

AGENT LEVEL

$$ mass = N(100, 20)$$
$$ \text{base-output} = N(100, 5)$$

ENVIRONMENT LEVEL

Note: $\Gamma(\alpha, \lambda)$ is a Gamma distribution with
$\alpha = \frac{\text{mean} \cdot \text{mean}}{\text{variance}}$
$\lambda = \frac{1}{\frac{\text{variance}}{\text{mean}}}$

$$ \text{recreation-activity} = \Gamma(0.5, 0.10) $$
$$ \text{domestic-activity} = \Gamma(0.5, 0.10) $$
$$ \text{transport-activity} = \Gamma(0.5, 0.10) $$
$$ \text{education-activity} = \Gamma(0.5, 0.10) $$

$$ \text{food-exposure} = \Gamma(0.5, 0.10) $$
$$ \text{food-energy-density} = \Gamma(0.5, 0.10) $$
$$ \text{food-advertising} = \Gamma(0.5, 0.10) $$
$$ \text{food-convenience} = \Gamma(0.5, 0.10) $$
$$ \text{food-literacy} = \Gamma(0.5, 0.10) $$





\end{document}
